\chapterwithauthor{Pradeepto Bhattacharya}{Building KDE's Community in India}

\authorbio{Pradeepto Bhattacharya started contributing to KDE in 2005. He started with KDE promotion in India and then contributed to KDE PIM for a few years. He founded the KDE India community and has served on KDE e.V.'s Board of Directors. He lives in Pune, India with his wife. He works for Red Hat in the Developer Tools Engineering Group.}

\noindent{}In March 2016, the KDE India community organized its 5th edition of conf.kde.in, the annual Qt/KDE conference in India, in Jaipur. Jaipur is a small laid-back city in the North West part of India. It took 10 years for KDE India to reach Jaipur. This essay tries to tell you that story briefly, the 10 years of KDE's journey in India, a sub-set of its own 20 years journey. A story in which many new friends were made who came together at different points of time to make this journey a very happy memory.

It started with a Birds of Feather session at FOSS.IN, at that point India's and perhaps Asia's largest Open Source conference, on December 2nd 2005. Two KDE developers, Till Adam and Sirtaj Singh Kang (Taj), were participating in the conference. I had flown down to Bangalore from Mumbai to attend my first ever Open Source conference. I used to write Qt code for a company in Mumbai and eagerly wanted to know what all this Open Source is about. The result of the BoF session was the founding of an informal group called KDE India. Very simple beginnings - a mailing list for the Indian community and a sub-domain in.kde.org because somebody owned kde.in already. Some time later, when that domain was available, KDE e.V took it under its wings. Aaron Seigo was supposed to attend the conference as well but he couldn't because of another clashing conference. He made it up by coming to India for FOSS.IN/2006.

After the conference in 2005, the community basically grew very slowly. I didn't let go of any speaking opportunity between 2006 and 2009. I went to any college or university or anyone organizing a FOSS event and would let me speak about KDE at their event.  Thanks to Google Summer of Code and Season of KDE, we got a few KDE contributors from India over the next few years. Earliest contributors that I can remember were Sharan Rao, Piyush Verma, Akarsh Simha etc. I am fortunate to be still in touch with all of them. In 2007, the FOSS.IN organizers came up with a brilliant master stroke - “project days” for various Open Source projects. These were modeled around LCA mini-confs or FOSDEM rooms. We had to submit proposals even for “project days”. KDE got a slot for a KDE Project Day. Long-time KDE contributors Kévin Ottens, Till Adam and Volker Krause all were present for the KDE Project Day and the main conference.

For any success story, there is always an inflection point. KDE India's story has one as well. It was in 2008. I contacted an acquaintance of mine, Madhusudhan CS, who was still an engineering student in Bangalore to help me with a certain idea for the KDE Project Day at FOSS.IN. I requested him to keep it a secret because I wasn't sure about the idea myself and how I would execute it. Anyway, Madhu seemed to be sold on the idea but he couldn't help himself from keeping the secret from his best friend and classmate Santosh Vattam. Both started helping with the idea the next day. As it turned out, Madhu and Santosh were excellent in selling the idea to their classmates and soon, I had a whole “secret” team - Krishna, Aditya - working on the idea. The idea was to create 1 or 2 KDE fliers and hand them out at the KDE Project Day. As it turned out, with the help of an excellent team, your idea can multiply manifold. The output wasn't just a couple of fliers but a complete booklet. The first KDE handbook was born. Soon we realized that we needed money to print the booklet and lots of it if we wanted to make it coloured and do it right. Taj came to rescue, he convinced his company to sponsor the complete costs - a princely sum of Rs. 40,000 - to get the book printed. Those booklets traveled to Jamaica, Nigeria, Europe, Malaysia, Taiwan and many other places.

The booklet wasn't the inflection point and Madhusudhan didn't just help me find volunteers for the project day. He also played the crucial role of introducing me to a junior student from his college - Shantanu Jha. That changed everything for KDE India and me. I finally found the guy who was crazier and better than I could ever dream of. It was the beginning of a great partnership for many years. I probably haven't trusted anyone as much in my life as Shantanu. Knowing that I could delegate work to him and it would get done was such a relief. With his help we kept organizing KDE Project Days and other co-opted events at various conferences across India. During this period, many contributors kept coming and going.

Shantanu followed the tradition of his mentors. He got more friends to contribute to KDE - Sinny (whom Shantanu married later), Sudhendhu. In March 2011, we organized the first conf.kde.in in Bangalore. A lot of sacrifices went into it by everyone. I don't think it would have been possible if anyone would have been missing in the team. A lot of new faces met for the first time - Vishesh, Rohan, Kunal, Smit etc. Since 2013, conf.kde.in has become an annual traveling event. The event has traveled from Bangalore to Gandhinagar to Kollam to Jaipur. After Bangalore in 2011, every year the event has been organized in a smaller city or town and it has always been organized in some university, a tradition that the Indian community has learned from Akademy.

In this past decade many people have played a crucial role in keeping the KDE Community up and running in India. Atul Chitnis, Kishore Bhargava, Tarique Sani, Swati Sani, Prashanth Udupa, Atul Jha, Sujith, Abhishek Patil, Supreeth Vattam, Kamal, Anurag, Aanjhan, Runa, Sankarshan, Kushal, Harish, Yash, Devaja, Akshay, Ashish, Shivam, Sagar. The list is long but not complete. Over the period of a decade, many KDE members from the US and Europe have visited India during the events and graced us with their knowledge and love. Adriaan de Groot, Sebastian K\"{u}gler, Lydia Pintscher, Simon Hausmann, Frederik Gladhorn, Kenny Duffus, Jos Poortvliet and others - you all are a part of the KDE India family.

A community always needs fresh and young blood to keep going strong. Slowly but surely we see new members join the KDE India community. The community is small but it keeps going on. Bhushan, Boudhayan and others contribute to KDE and are role models for the future generations of KDE contributors in India. The KDE family is growing and I really hope it keeps going.

I have seen this community or should I say this family and its journey first hand through my eyes and I have been a part of it. I am extremely proud of it and I feel humbled that I have had the opportunity to meet such a great set of people from across my country and the whole world. I feel I am really fortunate that I am a part of this journey and this family.
